Laser Tag is a light based electronic sport, popular amongst juveniles and enjoyed by persons of all ages. The technology used in laser tag sports systems has not seen much innovation over the years with most systems still requiring large cumbersome vests and a specialized arena.

This investigation focuses on the central component to laser tag systems, the uni-directional communication via a narrow beam of IR light. This poses as a unique engineering problem due to the high directionality requirement along with size, weight and cost constraints.

This project approaches the engineering problem by it breaking down into fine grained sub-problems. For each of these problems a modular component is developed and evaluated. Through the development and analysis of these modules a deep understanding of the tagger-target system is documented and areas of focus are identified for future investigation and development are proposed.