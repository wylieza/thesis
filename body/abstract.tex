%NB: this is going to be read through!
Laser Tag is a light-based electronic sport, popular amongst adolescents and enjoyed by persons of all ages. The technology used in laser tag sports systems has not seen much innovation over the years with most systems still requiring large cumbersome vests and a specialized arena.

This investigation focuses on the central component of laser tag systems, the uni-directional communication via a narrow beam of IR light. This poses a unique engineering design problem due to the high directionality requirement along with size, weight, and cost constraints.

This project approaches the engineering problem by it breaking down into fine-grained sub-problems, designing solutions for these sub-problems, and finally implementing and analysing the performance of these modules and the system as a whole.

During the course of this investigation an infrared-based line-of-sight communication system is designed, realized and analysed. The investigation provides insight into modularized components of such a system through experimentation and thorough analysis.

It is shown in this investigation that with a limited budget, it is possible to build a tagger-target system with a significant range and the ability to function in bright ambient conditions. In addition, various aspects of the developed system are critically analysed and suggestions for further research and development are proposed.


\iffalse
Your abstract provides a good idea of where the project is going, however, it's missing your key findings and conclusion. An abstract is a highly condensed summary of the entire project and usually follows the following format: intro and problem, project aim, methods, key findings and conclusion.

The abstract is usually the most difficult part to write because of how condensed it needs to be. If you need inspiration, have a look at the abstracts of relevant journal articles :)
\fi