%NB: this is definitely going to be read through!
Laser Tag is a light-based electronic sport, popular amongst juveniles and enjoyed by persons of all ages. The technology used in laser tag sports systems has not seen much innovation over the years with most systems still requiring large cumbersome vests and a specialized arena.

This investigation focuses on the central component of laser tag systems, the uni-directional communication via a narrow beam of IR light. This poses a unique engineering design problem due to the high directionality requirement along with size, weight, and cost constraints.

This project approaches the engineering problem by it breaking down into fine-grained sub-problems, designing solutions for these sub-problems, and finally implementing and analysing the performance of these modules and the system as a whole.

%todo: improve this paragraph
At the centre of this project is the design, realization and analysis of an infrared based line-of-sight communication system. This investigation aims to provide insight into the components of such a system through the thorough analysis of modules as well as document what kind of performance can be achieved on a limited budget in terms of range and tolerance to ambient lighting.



%A breakdown of the project methodology follows the literature review. The design phase is then documented, during which subsystems are identified and designed. Following the design phase, the modules are realized and experimentally evaluated. During the final phases of this project the results are analysed and conclusions are drawn.
