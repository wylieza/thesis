%NB: this is definitely going to be read through!
Laser Tag is a light-based electronic sport, popular amongst juveniles and enjoyed by persons of all ages. The technology used in laser tag sports systems has not seen much innovation over the years with most systems still requiring large cumbersome vests and a specialized arena.

This investigation focuses on the central component of laser tag systems, the uni-directional communication via a narrow beam of IR light. This poses a unique engineering design problem due to the high directionality requirement along with size, weight, and cost constraints.

This project approaches the engineering problem by it breaking down into fine-grained sub-problems, designing solutions for these sub-problems, and finally implementing and analysing the performance of these modules and the system as a whole.

%todo: improve this paragraph
The investigation begins with an introduction to the problem and a review of the relevant literature. Following the literature review is a break down of the project methodology. In the design phase, the subsystems are identified and designed. The experimentation chapter then specifies the performance test procedures. The results are then analysed and conclusions are drawn.
