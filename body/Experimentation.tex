\chapter{Experimentation}
\label{ch_experimentation}
%This is a chapter added into the template because Dr. Winberg's example reports have it.

\section{Testing Methodology}

Testing was performed in two phases, the first phase involves testing of individual modules. This modular testing is analogous to unit-tests in the computer programming paradigm. The second phase of testing involves characterising the performance of the final system prototype.

\subsection{LED Driver}
The following tests were performed...

\subsection{Tests $\setminus$ Results to gather}
\textbf{Hardware}
\begin{itemize}
	\item IR Power LED:
	\begin{itemize}
		\item Beam angle (focus vs no focus)
		\item Beam strength (LUX)
		\item LED Current draw
		\item LED temperature (basic sensor or a temperature gun that I borrow from somewhere)
		\item MOSFET temperature
	\end{itemize}
	\item For each photo-sensor (photodiode, phototransistor, all in one receiver module):
	\begin{itemize}
		\item Receiving beam-angle
		\item Output signal strength vs light-intensity/distance
		\item Reaction time (rise time)
		\item multipath/picking up reflections
	\end{itemize}
	\item Op-Amp:
	\begin{itemize}
		\item Op-amp gain performance (push to limit) (at 36kHz)
		\item Op-amp frequency performance (at some fixed gain)
		\item Filter performance (anti-aliasing and high pass filtering)
	\end{itemize}
\end{itemize}

\textbf{Software}
\begin{itemize}
	
	\item Transmitter
	\begin{itemize}
		\item Timing accuracy
	\end{itemize}
	
	%\item Protocol
	%\begin{itemize}
	%	\item
	%\end{itemize}
	
	\item Receiver
	\begin{itemize}
		\item Bit error rate / max receiver distance - (not sure how I will implement this yet)
		\item Bit error correction - (not sure how I might implement this yet)
		\item Modulation to High/Low logic value [DSP processing analysis]
		\begin{itemize}
			\item Latency
			\item Testing the timing between stages of demodulating and decoding
		\end{itemize}
	\end{itemize}
	
\end{itemize}