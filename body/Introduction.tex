\chapter{Introduction}
\label{ch_introduction}

\section{Background to the study}
%A very brief background to your area of research. Start off with a general introduction to the area and
%then narrow it down to your focus area. Used to set the scene \cite{Knight2013}.
Since the discovery of infrared (IR) light by William Herschel in the 1800s \cite{Rowan-Robinson2013}, it has found application in technologies ranging from military grade night vision equipment to the humble television remote.

On the electromagnetic spectrum infrared radiation is just below visible light with a slightly longer wavelength. Being invisible to the naked eye while retaining much the same behavioural properties as visible light is one of the reasons it has so many useful applications.

This study focuses on the application of IR in the recreational sport of laser tag. Although many use cases of infrared involve passive observation, this investigation seeks to understand appropriate means to transmit and receive information by means of a narrow infrared beam for the purposes of unidirectional communication.

When it comes to laser tag, there is commercial grade equipment that one my use at laser tag companies. However when one considers the consumer market there are noticeable differences between the more sophisticated commercial equipment and the cheaper laser tag toys available on the market. There is very little research and documentation on the creation of these laser tag systems.

There laser tag systems operate on many of the same principles as television remote controls. A topic on which much is known about the use of infrared in remote controls and there is a wealth of research, documentation and tutorials on the subject. Laser tag systems is in essence a specialized form of the television remote and much less information is available on the specialized use of infrared in the laser tag context. There are several factors and constraints which are unique in this context.



\section{Objectives of this study}

The primary objective of this study is to develop the modules for a uni-directional data link (UDDL) between a tagger and a target in a laser tag system. There are a vast number of techniques that could be employed to achieve this goal. This study aims to select appropriate techniques and implement these to allow their performance to be empirically tested. This study aims to serve as a platform for further research by performing an initial investigation into a UDDL system in the context of laser tag.

The uni-directional data link system (UDDLS) is complex, comprising of many independent functional components. For the purposes of this investigation, these functional components will be separated into modules. Modular design allows for more fine grained testing and experimentation which in turn allows the performance of individual components to be evaluated.

In addition to the primary objective, which is to develop a practical UDDL in the context of laser tag. This study also aims to compare the effectiveness of different substitute modules. Specifically, this study aims to compare the use of a photodiode and phototransistor for the purposes of detecting IR light.

Finally, there exist IR receiver chips (used in standard IR remote control applications) which perform a large portion of the work done by the receiver side of such a laser tag system inside a single package. As a control for the investigation, this study will use one of these as a golden measure with which to compare the full system.

\subsection{Problems to be investigated}
%Description of the main questions to be investigated in this study...

At the highest abstraction level, the first step of this investigation is to determine what modules are required to develop the UDDLS. This will require a breakdown of the fundamental system requirements into functional blocks.

After the system has been effectively sub-divided into appropriate functional blocks, techniques for implementing each block in the form of an independent module must be decided. The technique used to implement the functional block is determined based on the available equipment and resources and should be in line with standard practices.

Once a module has been designed and implemented, various tests specific to that modules function should be performed to determine that the module is able to perform its function and to test the limits to the modules functionality.

As hilighted in the objectives of this study, certain modules with the same function will be designed to provide a comparison between the performance of a photodiode and a phototransistor for the purposes of incident IR light measurements.

This investigation may be summarised through the following questions:

\begin{itemize}
	\item What functional blocks are required to implement the tagger-target data link system?
	\item How might these functional blocks be realized?
	\item What is the predicted performance of these designs?
	\item What is the measured performance of the individual modules and how does the system perform as a whole?
	\item How do the different modules for the detection of incident IR light compare?
%	
%	\item What techniques are exist for the generation of narrow beam IR radiation?
%	\item What is a suitable protocol for the tagger/target system to utilize for detecting and identifying shots fired by opponents?
%	\item What light detection technology is the most appropriate in the context of laser tag?
%	\item What kind of performance does one expect from such a system?
\end{itemize}


\iffalse

-----------------------------------------------------

This study must briefly map out the various functional blocks required for a laser tag system. Following this functional blocks should be realised, each theoretical block will translate to a single module. Modules will consist of some combination of software and hardware components. This combination will be based of the appropriate technique decided on.

An appropriate protocol for encoding information into an infrared beam needs to be developed. An investigation into the current industry-standard protocols will be performed and an appropriate protocol selected and modified to satisfy problem constraints.

An instigation into the process of generating a narrow angle beam of IR is to be investigated. This is to ensure that a measure of skill and accuracy is required in order for the player to hit the target. Techniques will be evaluated on their cost, robustness and size.

An appropriate sensor module must be developed to act as a wide angle receiver. Choosing an appropriate photo diode for this module will involve a comparison of various available options. Experiments will be performed to determine the cost vs performance of these sensors.

The investigation will also broadly cover the selection of an appropriate microcontroller and the design of various modules which will be combined into the experimental setup.

----------------------------------------

Laser Tag become available in the commercial industry in the 1980's. It is understood that the inventor of commercial laser tag is George Carter who founded Photon, the first large scale a laser tag franchise that provided an arena and laser tag equipment for it's members to participate in laser tag games. Carter is said to have been inspired by his love for Star Wars.

Laser tag has had varying periods of popularity.
There are several locations in Cape Town which facilitate laser tag games.

Commercial grade laser tag systems consist of elaborate vests which provide ample target area and the laser guns (phasors) have sophisticated IR emitters. However when it comes to personal laser tag equipment the technology is far less advanced and leaves one wanting.

In addition to the more toy like style of consumer grade laser tag equipment, there are  Of course over the years there have been consumer grade toys that enter the market, these are much more available in the overseas markets such as the USA.


\fi

\subsection{Purpose of the study}
%Give the significance of investigating these problems. It must be obvious why you are doing this study
%and why it is relevant.

%The purpose of this study is therefore to investigate the critical components of such a system, with the goal of providing a comprehensive understanding of how the various block of such a system can be most efficiently brought together and optimized.

%------------------

The purpose of this study is to investigate the design and construction of a uni-directional data link in the context of laser tag system. This investigation will attempt to determine the effectiveness and measure performance of different key components from which the tagger and target units comprise.

In doing so this study aims to make available insightful information in the context of laser tag systems. Because of the niche nature of this system in the field of recreational electronics, there are very few sources of information available that formally document the development of a recreational laser tag system.

By performing a detailed investigation into the performance of the individual modules of the UDDL system, this study hopes to provide a foundation for further research into the individual modules and over-all system in an effort to overcome any short-comings and improve on the results obtained in this study.

In summary, this study aims to achieve the following:

\begin{itemize}
	\item Demonstrate how one might create the UDDL system in the context of a laser tag system
	\item Provide insight into the complications and limitations of such a system
	\item Provide a foundation for further research and development of such a system or particular subsystems.
\end{itemize}


\section{Scope and Limitations}
%Scope indicates to the reader what has and has not been included in the study. Limitations tell the
%reader what factors influenced the study such as sample size, time etc. It is not a section for excuses as
%to why your project may or may not have worked.

This investigation approaches the problems to be investigated at a functional level and therefore does not concern the design of any casing or aesthetic assembly. This limitation is to ensure the study remains focused. As highlighted in the previous section, the purpose is to gain an understanding of the key modules and their respective performance, not to design or create 'ready to play' laser tag equipment.

The completion of this project will involve the design and realization of a tagger and target module for the purposes of performing tests. All final implementations will be done on strip-board as opposed to PCB, this is due to the Covid-19 pandemic which has made global markets less accessible and time constraints do not allow for the possibility of unpredictable delays.

Although no time will be allocated to the design aesthetics or 'form' component of the laser tag system that this investigation is focused around. Consideration will be taken with regard to the constraints such a system puts on the engineering design decisions. That is to say that all modules will be designed with size, weight and power constraints in mind, even though such constraints do not directly influence what is possible during the course of this investigation.

This project has a budget of R1500 for the procurement of components. In addition to the limited amount of time allocated to this investigation, this constraint has limited the number of components that can be compared. This limitation has also ruled out the use of IR laser diodes, which although in themselves fit within the budget, would require a myriad of personal protective equipment and specialized lenses which is beyond the budget of this investigation.

No automatic gain control considered

No error correction





\section{Plan of development}
Here you tell the reader how your report has been organised and what is included in each
chapter.

{\bf I recommend that you write this section last. You can then tailor it to your report.}
