\chapter{Introduction}

\section{Background to the study}
%A very brief background to your area of research. Start off with a general introduction to the area and
%then narrow it down to your focus area. Used to set the scene \cite{Knight2013}.
Since the discovery of infrared (IR) light by William Herschel in the 1800s \cite{Rowan-Robinson2013}, it has found application in technologies ranging from military grade night vision equipment to the humble television remote.

On the electromagnetic spectrum infrared radiation is just below visible light with a slightly longer wavelength. Being invisible to the naked eye while retaining much the same behavioural properties as visible light is one of the reasons it has so many useful applications.

This study focuses on the application of IR in the recreational sport of laser tag. Although many use cases of infrared involve passive observation, this investigation seeks to understand appropriate means to generate and detect a narrow infrared beam for the purposes of unidirectional communication.

Much is known about the use of infrared in remote controls, but less information is available on the specialized use of infrared in the laser tag context. This introduces several factors and constraints which are unpacked and addressed.



\section{Objectives of this study}
\subsection{Problems to be investigated}
%Description of the main questions to be investigated in this study...
An appropriate protocol for encoding information into an infrared beam needs to be developed. An investigation into the current industry-standard protocols will be performed and an appropriate protocol selected and modified to satisfy problem constraints.

An instigation into the process of generating a narrow angle beam of IR is to be investigated. This is to ensure that a measure of skill and accuracy is required in order for the player to hit the target. Techniques will be evaluated on their cost, robustness and size.

An appropriate sensor module must be developed to act as a wide angle receiver. Choosing an appropriate photo diode for this module will involve a comparison of various available options. Experiments will be performed to determine the cost vs performance of these sensors.

The investigation will also broadly cover the selection of an appropriate microcontroller and the design of various modules which will be combined into the experimental setup.

\subsection{Purpose of the study}
%Give the significance of investigating these problems. It must be obvious why you are doing this study
%and why it is relevant.
Because of the niche nature of this application specific investigation, there is little to no sources of information available that formally document the development of a recreational laser tag system.

The purpose of this study is therefore to investigate the critical components of such a system, with the goal of providing a comprehensive understanding of how the various block of such a system can be most efficiently brought together and optimized. 

\section{Scope and Limitations}
Scope indicates to the reader what has and has not been included in the study. Limitations tell the
reader what factors influenced the study such as sample size, time etc. It is not a section for excuses as
to why your project may or may not have worked.

\section{Plan of development}
Here you tell the reader how your report has been organised and what is included in each
chapter.

{\bf I recommend that you write this section last. You can then tailor it to your report.}
