\chapter{Introduction}
\label{ch_introduction}

This research project aims to investigate the tagger-target engineering problem in the context of laser tag.  This involves decomposing high-level systems into sub-systems, developing hardware and software modules for these sub-systems and finally evaluating the performance of these sub-systems.

In this study, the relevant literature is used to develop a tagger-target system within the time and budget constraints. Throughout this investigation a variety of engineering problems are encountered and addressed.

It is intended for the knowledge generated throughout this investigation to contribute toward the further development of laser tag systems.

\section{Background to the study}
%A very brief background in your area of research. Start with a general introduction to the area and
%then narrow it down to your focus area. Used to set the scene \cite{Knight2013}.
Since the discovery of infrared (IR) light by William Herschel in the 1800s \cite{Rowan-Robinson2013}, it has found application in technologies ranging from military-grade night vision equipment to the television remote.

Inspired by star wars, the recreational sport of laser tag was allegedly invented by George Carter who patented his idea in the 1980s\cite{Carter1986}. In laser tag, each player is equipped with a tagger which may be used to fire a beam of infrared light at an opponent and a set of infrared sensing targets which are placed around the body to register incoming shots.

Although laser tag equipment has been used in commercial arenas for many decades, the equipment is still bulky, restricted to low-light conditions and only operational over short ranges. Laser tag toys have also been available for several decades, although these are not as cumbersome they are less sophisticated when compared to their commercial grade counterparts.

Technology for generating and sensing infrared light has seen steady improvements over the decades and the cost of analog and digital signal processing devices has steadily declined. It seems there is much potential for the advancement and improvement of laser tag systems through the incorporation of these technological advancements.

\section{Objectives of this study}

The primary objective of this study is to develop the modules for a uni-directional data link (UDDL) between a tagger and a target in a laser tag system. There are a vast number of techniques that could be employed to achieve this goal. This study aims to select a few appropriate techniques and implement these to empirically analyse their performance. This study aims to serve as a platform for further research by performing an initial investigation into a UDDL system in the context of laser tag.

The uni-directional data link system is complex, comprising of many independent functional components. For this investigation, these functional components will be separated into modules. Modular design allows for more fine-grained testing and experimentation which in turn allows the performance of individual components to be evaluated.

In addition to the primary objective, which is to develop a practical data link in the context of laser tag. This study also aims to compare the effectiveness of different substitute modules. Specifically, this study aims to compare the use of a photodiode and phototransistor to detect IR light.

Finally, there exist IR receiver chips (used in standard IR remote control applications) which perform a large portion of the work done by the receiver side of such a laser tag system inside a single package. As a control for the investigation, this study will use one of these as a golden measure with which to compare the full system.

\subsection{Problems to be investigated}
%Description of the main questions to be investigated in this study...

At the highest abstraction level, the first step of this investigation is to determine what modules are required to develop the UDDL. This will require a breakdown of the fundamental system requirements into functional blocks.

After the system has been effectively sub-divided into appropriate functional blocks, techniques for implementing each block in the form of an independent module must be decided. The technique used to implement the functional block is determined based on the available equipment and resources and should be in line with standard practices.

Once a module has been designed and implemented, various tests specific to the function of that module should be performed to determine that the module can perform its function and to test the limits to the functionality of the module.

As highlighted in the objectives of this study, certain modules with the same function will be designed to provide a comparison between the performance of a photodiode and a phototransistor for incident IR light measurements.

This investigation may be summarised through the following questions:

\begin{itemize}
	\item What fundamental blocks are required to implement the tagger-target data link system?
	\item How might these fundamental blocks be realized?
	\item What is the predicted performance of these designs?
	\item What is the measured performance of the individual modules and how does the system perform as a whole?
	\item How do the different modules for the detection of incident IR light compare?
	\item How does this system compare to all in one IR receiver devices
	\item How might the RC-5 protocol be adapted for laser tag?
	
	%	
	%	\item What techniques exist for the generation of narrow beam IR radiation?
	%	\item What is a suitable protocol for the tagger/target system to utilize for detecting and identifying shots fired by opponents?
	%	\item What light detection technology is the most appropriate in the context of laser tag?
	%	\item What kind of performance does one expect from such a system?
\end{itemize}


\subsection{Purpose of the study}
%Give the significance of investigating these problems. It must be obvious why you are doing this study
%and why it is relevant.

%The purpose of this study is therefore to investigate the critical components of such a system, to provide a comprehensive understanding of how the various block of such a system can be most efficiently brought together and optimized.

%------------------

The purpose of this study is to investigate the design and construction of a uni-directional data link in the context of a laser tag system. This investigation will attempt to determine the effectiveness and measure performance of different key components from which the tagger and target units comprise.

In doing so this study aims to make available insightful information in the context of laser tag systems. Because of the niche nature of this system in the field of recreational electronics, there are very few sources of information available that formally document the development of a recreational laser tag system.

By performing a detailed investigation into the performance of the individual modules of the UDDL system, this study hopes to provide a foundation for further research into the individual modules and over-all system to overcome any shortcomings and improve on the results obtained in this study.

In summary, this study aims to achieve the following:

\begin{itemize}
	\item Demonstrate how one might create the uni-directional data link in the context of a laser tag system
	\item Provide insight into the complications and limitations of such a system
	\item Provide a foundation for further research and development of such a system or particular subsystems.
\end{itemize}


\section{Scope and Limitations}
%Scope indicates to the reader what has and has not been included in the study. Limitations tell the
%reader what factors influenced the study such as sample size, time etc. It is not a section for excuses as
%to why your project may or may not have worked.

This investigation approaches the problems to be investigated at a functional level and therefore does not concern the design of any casing or aesthetic assembly. This limitation is to ensure the study remains focused. As highlighted in the previous section, the purpose is to gain an understanding of the key modules and their respective performance, not to design or create 'ready to play' laser tag equipment.

The completion of this project will involve the design and realization of a tagger and target module to perform tests. All final implementations will be done on strip-board as opposed to PCB, this is due to the Covid-19 pandemic which has made global markets less accessible and time constraints do not allow for the possibility of unpredictable delays.

Although no time will be allocated to the design aesthetics or 'form' component of the laser tag system that this investigation is focused around. Consideration of the contextual design constraints will be made. That is to say that all modules will be designed with size, weight and power constraints in mind, even though such constraints do not directly influence what is possible during the course of this investigation.

This project has a budget of R1500 for the procurement of components. In addition to the limited amount of time allocated to this investigation, this constraint has limited the number of components that can be compared. This limitation has also ruled out the use of IR laser diodes, which although in themselves fit within the budget, would require a myriad of personal protective equipment and specialized lenses which is beyond the budget of this investigation.

No automatic gain control considered (todo: elaborate)

No error correction but basic error detection (todo: elaborate)





\section{Plan of development}
%Here you tell the reader how your report has been organised and what is included in each chapter.
% {\bf I recommend that you write this section last. You can then tailor it to your report.}

%todo: this section