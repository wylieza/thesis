\chapter{Conclusions}
\label{ch_conclusions}

%These are the conclusions from the investigation and how the investigation changes things in this field or contributes to current knowledge...

%Draw suitable and intelligent conclusions from your results and subsequent discussion.

The core objective of this study was to develop and investigate a modular tagger-target system. Through the process of accomplishing this objective, insights into the potential for laser tag systems which are tolerant to ambient lighting conditions and work over longer distances have been revealed.
In addition to the core objective, this investigation set out to design and evaluate each module in the full system to provide a foundation for further research. This has also been achieved. The following chapter concludes this investigation with several closing remarks pertaining to the system as a whole as well as the various modules it comprises.

\section{Review of System}

Through the course of this investigation, a fully functional laser tag system was successfully implemented. The design for each module has been documented and various experiments have allowed the performance of these modules to be evaluated.

The four main problems presented in the introduction to this study have been addressed through the successful development and implementation of this modular tagger-target system. 

This study has shown that there is great potential for long-range laser tag systems which are capable of functioning in harsh lighting conditions. The range achieved using the system in its current state was only marginally better than that of commercial laser tag systems. However, the shortcomings of the system have been identified and the range could be drastically improved by following the recommendations presented at the end of this report.

\section{Review of Modules}
This following section provides an overview of the performance for each module within the greater context of the whole system. To do this the modules have been categorized into one of two categories.

%Those categorized as having excellent performance 



\subsection{Excellent Performance}

\textbf{Light Focus System}\\
The light focus system produced an optimal beam in the context of laser tag. The beam spot diameter at short distances (less than 30m) was less than 1m. This beam angle is the ideal size as it ensures some skill is required to aim the tagger while ensuring it isn't so challenging that it is impossible to make a single shot.

It was noted during the system test that for distances greater than 30m, the diameter of light detectable by the IR detectors was less than the size predicted by the beam dispersion experiment. This effect worked to prevent the tagger from being too easy to aim at distant targets.

\textbf{Power LED Driver}\\
The power LED driver module exhibited great performance. The linear current regulation resulted in very precise control over the timing and introduced very little noise into the system. Also, by using a supply with a voltage close to that of the LED, very little heat was dissipated by the driving circuitry.

\textbf{IR Phototransistor Detector}\\
The IR phototransistor detector outperformed the photodiode detector because the amplification stage was robust to saturation caused by a large DC offset.

\textbf{IR Receiver}\\
The all in one package IR receiver outperformed the other IR detector modules, this was likely due to the built-in automatic gain control stage.

\textbf{Tone Decoder}\\ %todo: I want to talk about the speed of detection, but didn't write about the experiment!!! - perhaps if there is the time in the end...
The digital tone decoder module exhibited exemplary performance despite the absence of dedicated DSP hardware units. The STM32F0 posed several challenges that had to be overcome before a working solution could be implemented, however once implemented the module was very effective.

The Goertzel algorithm was very quick to detect the presence of a tone and the Schmitt-trigger ensured the output was stable. This high performance was only possible through the optimization of the Goertzel algorithm.

The tone decoder did not handle small amplitude signals well, however, this is due to the absence of some form of gain control and not a problem with the tone decoder's algorithm.

\subsection{Adequate Performance}

\textbf{Tagger and Target MCUs}\\
The STM32F0 MCU was able to comfortably encode and decode Manchester transmissions. However, because the MCU did not feature a hardware floating-point unit or other hardware units useful for DSP, the tone decoding had to be implemented on a separate microcontroller. The use of two separate microcontrollers in the receiver is a sub-optimal solution.

%Similarly, the tagger MCU could not be configured to generate the 36kHz waveform in addition to performing the Manchester encoding. Therefore the implementation of a separate carrier waveform generation module was required.

\textbf{Carrier Waveform Generator}\\
The carrier waveform generation module was able to generate a high-frequency tone with a negligible delay in response to an input. However, the module did exhibit some undesirable behaviours.

During prolonged operation the 555 timer would heat up significantly, this was not a cause for serious concern however as the IC is usually only operated in short bursts during transmissions. It was noted that the frequency of the generated tone was sensitive to the temperature of the IC. This effect, undesirable, did not impact the performance of the overall system because the tone decoder had a large frequency tolerance. 

\textbf{Photodiode Detector}
The photodiode detector performed well in low light conditions, however, the transimpedance amplifier would saturate in bright ambient conditions. The photodiode has a slightly smaller angle of sensitivity when compared the photodiode.

\textbf{Signal Conditioning Modules}
The anti-aliasing filter performed sufficiently well and very little aliasing was observed. The precision rectifier introduces only a small amount of distortion into the rectified waveform.

%%%%%%%%%%%%%%%%%%%%















